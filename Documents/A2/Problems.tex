\section{
        \grayBox{\textcolor{purple}{APKALA}}
        \grayBox{\textcolor{blue}{}}
    }
    \RTL
        \subsection{\rl{کلاس} \lr{Product}}
    در اینجا کمی تبدیل فضای مسئله به فضای راه‌حل با استفاده از زبان شی‌گرای سی‌شارپ را تمرین می‌کنیم:
    
    کلاس محصول دارای ویژگی های 
    مولفه مشخصه(
     \grayBox{\textcolor{navyBlue}{Id}}
    )
    ،
    نام(
    \grayBox{\textcolor{navyBlue}{Name}}
    )
    ،
    قیمت(
    \grayBox{\textcolor{navyBlue}{Price}}
    )
    و 
    نمره از لحاظ محبوبیت(
    \grayBox{\textcolor{navyBlue}{Rate}}
    )
    است که یک محصول را از دیگری تمایز میدهد.
    
    
     ممکن است موجودیتی که در نظر می‌گیریم ویژگی(های) دیگری هم داشته باشد اما ما فقط ویژگی‌هایی را در نظر می‌گیریم که به حل مسئله کمک می‌کند.
    
    \textbf{گام اول:}
     برای هر 
    \grayBox{\textcolor{navyBlue}{property}}ای 
    که در کلاس وجود دارد 
    \grayBox{\textcolor{navyBlue}{getter}}
    و 
    \grayBox{\textcolor{navyBlue}{setter}}
    مناسب بنویسید.
    
    \textbf{گام دوم: }
    شما باید سازنده
    (\grayBox{\textcolor{navyBlue}{constructor}}) 
    این کلاس را تکمیل کنید تا شی‌ای که از کلاس ساخته می‌شود معتبر باشد یعنی هر 
    \grayBox{\textcolor{navyBlue}{Product}}ای 
    که ساخته می‌شود لزوما دارای ویژگی های ذکر شده باشد.
    
    پس از پیاده‌سازی صحیح سازنده‌ی این کلاس تستِ 
    \grayBox{\textcolor{dkgreen}{ProductConstructorTest}}
    پاس خواهد شد.
    
    بعد از پیاده‌سازی این کلاس و پاس شدن تست‌های آن کار شما با این کلاس و فایل آن تمام شده است. دیگر نیازی به تغییر این کد نخواهید داشت.
%---------------------------------------------------------------
    
    \subsection{\rl{کلاس} \lr{Categoty}}
    
    کلاس دسته بندی دارای ویژگی های 
    مولفه مشخصه(
     \grayBox{\textcolor{navyBlue}{Id}}
    )
    و 
    محصولات(
    \grayBox{\textcolor{navyBlue}{Products}}
    )
    است که یک دسته را از دیگری تمایز میدهد.
    برای مثال تمام گوشی های موبایل در یک دسته بندی قرار گیرند.
    \\
    \textbf{گام اول:}
     برای هر 
    \grayBox{\textcolor{navyBlue}{property}}ای 
    که در کلاس وجود دارد 
    \grayBox{\textcolor{navyBlue}{getter}}
    و 
    \grayBox{\textcolor{navyBlue}{setter}}
    مناسب بنویسید.
    \\
    \textbf{گام دوم: }
    شما باید سازنده
    (\grayBox{\textcolor{navyBlue}{constructor}}) 
    این کلاس را تکمیل کنید تا شی‌ای که از کلاس ساخته می‌شود معتبر باشد یعنی هر 
    \grayBox{\textcolor{navyBlue}{Category}}ای 
    که ساخته می‌شود لزوما دارای ویژگی های ذکر شده باشد.
    
    پس از پیاده‌سازی صحیح سازنده‌ی این کلاس تستِ 
    \grayBox{\textcolor{dkgreen}{CategoryConstructorTest}}
    پاس خواهد شد.
    \\
    \textbf{گام سوم: }
    شما باید متد 
    \grayBox{\textcolor{navyBlue}{AddProduct}}
    را بگونه ای پیاده سازی کنید که ورودی آن یک محصول باشد و با فراخوانی متد آن محصول را به محصولات خود اضافه کند.
    
    پس از پیاده‌سازی صحیح تستِ 
    \grayBox{\textcolor{dkgreen}{CategoryAddProductTest}}
    پاس خواهد شد.
\\
    \textbf{گام چهارم: }
    شما باید متد 
    \grayBox{\textcolor{navyBlue}{FilterByPrice}}
    را بگونه ای پیاده سازی کنید که دو ورودی از جنس 
    \grayBox{\textcolor{blue}{int}}
    بگیرد و تمامی محصولات آن دسته بندی، که قیمتشان بین دو عدد ورودی است را به عنوان خروجی بازگرداند.
    
    پس از پیاده‌سازی صحیح تستِ 
    \grayBox{\textcolor{dkgreen}{CategoryFilterByPriceTest}}
    پاس خواهد شد.
    
%---------------------------------------------------------------
    \subsection{\rl{کلاس} \lr{Cart}}
    کلاس سبد خرید دارای ویژگی های 
    نام صاحب سبد(
    \grayBox{\textcolor{navyBlue}{Owner}}
    )
    و 
    محصولات(
    \grayBox{\textcolor{navyBlue}{Products}}
    )
    است که یک دسته را از دیگری تمایز میدهد.
    \\
    \textbf{گام اول:}
     برای هر 
    \grayBox{\textcolor{navyBlue}{property}}ای 
    که در کلاس وجود دارد 
    \grayBox{\textcolor{navyBlue}{getter}}
    و 
    \grayBox{\textcolor{navyBlue}{setter}}
    مناسب بنویسید.
    \\
    \textbf{گام دوم: }
    شما باید سازنده
    (\grayBox{\textcolor{navyBlue}{constructor}}) 
    این کلاس را تکمیل کنید تا شی‌ای که از کلاس ساخته می‌شود معتبر باشد یعنی هر 
    \grayBox{\textcolor{navyBlue}{Cart}}ای 
    که ساخته می‌شود لزوما دارای ویژگی های ذکر شده باشد.
    
    پس از پیاده‌سازی صحیح سازنده‌ی این کلاس تستِ 
    \grayBox{\textcolor{dkgreen}{CartConstructorTest}}
    پاس خواهد شد.
    \\
    \textbf{گام سوم: }
    شما باید متد 
    \grayBox{\textcolor{navyBlue}{AddProduct}}
    را بگونه ای پیاده سازی کنید که ورودی آن یک محصول باشد و با فراخوانی متد آن محصول را به محصولات خود اضافه کند.
    
    پس از پیاده‌سازی صحیح تستِ 
    \grayBox{\textcolor{dkgreen}{CartAddProductTest}}
    پاس خواهد شد.
    \\
    \textbf{گام چهارم: }
    شما باید متد 
    \grayBox{\textcolor{navyBlue}{CalculatePrice}}
    را بگونه ای پیاده سازی کنید که 
    با فراخوانی آن 
    قیمت کل آن سبد خرید را بازگرداند.
    
    پس از پیاده‌سازی صحیح تستِ 
    \grayBox{\textcolor{dkgreen}{CartCalculatePriceTest}}
    پاس خواهد شد.
%---------------------------------------------------------------
    \subsection{\rl{کلاس} \lr{Store}}
    کلاس فروشگاه دارای ویژگی های 
    نام (
    \grayBox{\textcolor{navyBlue}{Name}}
    )
    ، 
    دسته بندی ها(
    \grayBox{\textcolor{navyBlue}{Categories}}
    )
    و
    سبد های خرید(
    \grayBox{\textcolor{navyBlue}{Carts}}
    )
    است که یک فروشگاه را از دیگری تمایز میدهد.
    \\
    \textbf{گام اول:}
     برای هر 
    \grayBox{\textcolor{navyBlue}{property}}ای 
    که در کلاس وجود دارد 
    \grayBox{\textcolor{navyBlue}{getter}}
    و 
    \grayBox{\textcolor{navyBlue}{setter}}
    مناسب بنویسید.
    \\
    \textbf{گام دوم: }
    شما باید سازنده
    (\grayBox{\textcolor{navyBlue}{constructor}}) 
    این کلاس را تکمیل کنید تا شی‌ای که از کلاس ساخته می‌شود معتبر باشد یعنی هر 
    \grayBox{\textcolor{navyBlue}{Store}}ای 
    که ساخته می‌شود لزوما دارای ویژگی های ذکر شده باشد.
    
    پس از پیاده‌سازی صحیح سازنده‌ی این کلاس تستِ 
    \grayBox{\textcolor{dkgreen}{StoreConstructorTest}}
    پاس خواهد شد.
    \\
    \textbf{گام سوم: }
    شما باید متد 
    \grayBox{\textcolor{navyBlue}{AddCart}}
    را بگونه ای پیاده سازی کنید که ورودی آن یک سبد خرید باشد و با فراخوانی متد آن سبد خرید را به لیست سبد های خود اضافه کند.
    
    پس از پیاده‌سازی صحیح تستِ 
    \grayBox{\textcolor{dkgreen}{StoreAddCartTest}}
    پاس خواهد شد.
    \\
    \textbf{گام چهارم: }
    شما باید متد 
    \grayBox{\textcolor{navyBlue}{AddCategory}}
    را بگونه ای پیاده سازی کنید که 
    ورودی آن یک دسته بندی باشد و با فراخوانی متد، آن دسته بندی را به لیست سبد های خود اضافه کند.
    
    پس از پیاده‌سازی صحیح تستِ 
    \grayBox{\textcolor{dkgreen}{StoreAddCategoryTest}}
    پاس خواهد شد.
    \\
    \textbf{گام پنجم: }
    شما باید متد 
    \grayBox{\textcolor{navyBlue}{Bestselling}}
    را به گونه ای پیاده سازی کنید که 
    با فراخوانی آن، محصولی که از همه بیشتر به فروش رفته را بازگرداند.
    
    پس از پیاده‌سازی صحیح تستِ 
    \grayBox{\textcolor{dkgreen}{StoreBestsellingTest}}
    پاس خواهد شد.
    \\
    \textbf{گام ششم: }
    شما باید متد 
    \grayBox{\textcolor{navyBlue}{MostPopular}}
    را به گونه ای پیاده سازی کنید که 
    با فراخوانی آن، محصولی که از همه محبوب تر است، را بازگرداند.
    
    پس از پیاده‌سازی صحیح تستِ 
    \grayBox{\textcolor{dkgreen}{StoreMostPopularTest}}
    پاس خواهد شد.
\LTR


%---------------------------------------------------------------
\RTL
\newpage
    \section{
        \grayBox{\textcolor{purple}{Telegram}}
        \grayBox{\textcolor{blue}{}}
    }
    \RTL
 
    \subsection{\rl{کلاس} \lr{Person}}
    
    این کلاس دارای ویژگی های 
    مولفه مشخصه (
    \grayBox{\textcolor{navyBlue}{ID}}
    )
    ، 
    نام(
    \grayBox{\textcolor{navyBlue}{Name}}
    )
    ،
    مخاطبین(
    \grayBox{\textcolor{navyBlue}{Contacts}}
    )
    و 
    چت(
    \grayBox{\textcolor{navyBlue}{Chats}}
    ) است.
    لازم به ذکر است که هر عضو، مولفه مشخصه ی منحصر به فردی دارد که او را با بقیه اعضا تمایز میدهد.
    
    \RTL
    
    
    \textbf{گام اول: }
    شما باید سازنده
    (\grayBox{\textcolor{navyBlue}{constructor}}) 
    این کلاس را تکمیل کنید تا شی‌ای که از کلاس ساخته می‌شود معتبر باشد یعنی هر 
    \grayBox{\textcolor{navyBlue}{Person}}ای 
    که ساخته می‌شود لزوما دارای ویژگی های ذکر شده باشد.
    
    پس از پیاده‌سازی صحیح سازنده‌ی این کلاس تستِ 
    \grayBox{\textcolor{dkgreen}{test\_PersonConstructor}}
    پاس خواهد شد.
    \\
    \textbf{گام دوم: }
    متد 
    \grayBox{\textcolor{navyBlue}{AddContact}}
    را به گونه ای پیاده سازی کنید که 
    یک عضو(Person) بگیرد،
    و  اگر با آن عضو قبلاً مخاطب نبوده، آن عضو را به مخاطبین خود اضافه کند.
    
    
    پس از پیاده‌سازی صحیح تستِ 
    \grayBox{\textcolor{dkgreen}{test\_PersonAddContact}}
    پاس خواهد شد.
    
    \RTL
 
    \subsection{\rl{کلاس} \lr{Message}}
    
    این کلاس دارای ویژگی های 
    مبدا (
    \grayBox{\textcolor{navyBlue}{Source}}
    )
    ، 
    مقصد(
    \grayBox{\textcolor{navyBlue}{Destination}}
    )
    و
    متن پیام(
    \grayBox{\textcolor{navyBlue}{Context}}
    )
    است.
    
    
    \textbf{گام اول: }
    شما باید سازنده
    (\grayBox{\textcolor{navyBlue}{constructor}}) 
    این کلاس را تکمیل کنید تا شی‌ای که از کلاس ساخته می‌شود معتبر باشد یعنی هر 
    \grayBox{\textcolor{navyBlue}{Message}}ای 
    که ساخته می‌شود لزوما دارای ویژگی های ذکر شده باشد.
    
    پس از پیاده‌سازی صحیح سازنده‌ی این کلاس تستِ 
    \grayBox{\textcolor{dkgreen}{test\_MessageConstructor}}
    پاس خواهد شد.

    \subsection{\rl{ارتباط دو کلاس} }
    شما باید متد
    \grayBox{\textcolor{navyBlue}{SendMessage}}
    را برای کلاس Person
    را بگونه ای پیاده سازی کنید 
    که با گرفتن یک پیام(Message)
    ،
    به چت هر دو نفر، آن پیام را اضافه کند.
    همچنین اگر دو نفر با یکدیگر مخاطب نبودند
،   
ابتدا
    یکدیگر را به مخاطبین خود اضافه کنند.
     
     
    پس از پیاده‌سازی صحیح تستِ 
    \grayBox{\textcolor{dkgreen}{test\_SendMessage}}
    پاس خواهد شد.