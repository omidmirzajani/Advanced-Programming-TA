
    \section{
    مسائل بامزه
    \grayBox{\textcolor{purple}{C\#}}
    }
    همانطور که 
    \textbf{دیمو}
    \footnote{Dimo}
    را میشناسید، به تازگی علاقه مند به صنعت کامپیوتر شده است و مانند بقیه برای پیشرفت اوضاع درسی خود، در حال تمرین برنامه نویسی است. به او کمک کنید تا تکلیف خود را انجام دهد.
    \subsection{
    Reversing
    }
        متد
        \grayBox{\textcolor{blue}{Reverse}}
        را در کلاس Program
         به گونه ای پیاده سازی کنید، که یک آرایه بگیرد و آن را برعکس کند.
        به طور مثال اگر ورودی شامل اعداد (1و4و5و2و3) باشد، خروجی باید (3و2و5و4و1) باشد.
        دقت کنید که این متد باید برای همه انواع داده ای کار کند.
        \footnote{Method Generic}
        \\
        پس از پیاده سازی صحیح تست های 
        \grayBox{\textcolor{dkgreen}{ReverseIntTest}}
        و
        \grayBox{\textcolor{dkgreen}{ReverseStringTest}}
        پاس خواهند شد.
        
    \subsection{Class / Interface
    }
        \subsubsection{IPlayer}
        اینترفیس 
        IPlayer
        را به گونه ای پیاده سازی کنید که شامل ویژگی های 
        \begin{itemize}
            \item \grayBox{height} از نوع \grayBox{\textcolor{blue}{bool}}
            \item \grayBox{speed} از نوع \grayBox{\textcolor{blue}{bool}}
        \end{itemize}
        و متد
        \grayBox{Post()}
        با نوع خروجی 
        \grayBox{\textcolor{blue}{string}}
        باشد.
        \\
        پس از پیاده سازی صحیح تست 
        \grayBox{\textcolor{dkgreen}{IPlayerTest}}
        پاس خواهند شد.
        
        \subsubsection{Athlete}
        کلاس 
        Athlete
        را به گونه ای پیاده سازی کنید که اولاً یک Player باشد، یعنی ویژگی ها و متدهای IPlayer را دارا باشد.
        و هم چنین علاوه بر آن دارای یک  ویژگی به نام 
        \grayBox{name}
        از نوع
        \grayBox{\textcolor{blue}{string}}
        و متد
        \grayBox{Post()}
        با نوع خروجی 
        \grayBox{\textcolor{blue}{string}}
        باشد.
        \\
        ابتدا سازنده این کلاس را تکمیل کنید و با پیاده سازی صحیح آن، تست 
        \grayBox{\textcolor{dkgreen}{AthleteCostructorTest}}
        پاس خواهد شد.
        \\
        سپس متد 
        \grayBox{Post()}
        را به گونه ای پیاده سازی کنید که مشخص کنید این ورزشکار برای چه ورزشی مناسب است.
        \begin{itemize}
            \item اگر فرد قد بلند و سرعت بالایی داشت، برای بسکتبال مناسب است.
            
            \item اگر فرد فقط قد بلندی داشت، برای والیبال مناسب است.
            
            \item اگر فرد فقط سرعت بالایی داشت، برای دویدن مناسب است.
            
            \item اگر فرد هیچ یک از دو ویژگی ذکر شده را نداشت، برای کشتی مناسب است.
        \end{itemize}
        هم چنین دقت کنید که خروجی داده شده باید به صورتِ
        نامِ فرد در ابتدای رشته و سپس عبارت 
        \grayBox{is appropriate for}
        و سپس رشته ورزشی مناسب آن شخص باشد.
        پس از پیاده سازی صحیح تست 
        \grayBox{\textcolor{dkgreen}{AthletePost}}
        پاس خواهند شد.
    \subsection{Exceptions}
    خواندن متون فینگیلیش یکی از معضلات دیمو شده است و میخواهد کلماتی که دارای حروف صدادار نیستد را شناسایی کند.
    \\
    متد 
    \grayBox{Vowels}
    را در کلاس Program
    به گونه ای پیاده سازی کنید
    که یک ورودی از نوع رشته بگیرد و اگر آن رشته شامل حروف صدادار بود، استثنایی از نوع 
    \grayBox{FormatException}
    پرتاب کند. در غیر این صورت رشته 
    \grayBox{ Not Found!}
    را به عنوان خروجی بازگرداند.
    \\
    پس از پیاده سازی صحیح، تست های 
    \grayBox{\textcolor{dkgreen}{VowelSound}}
    و
    \grayBox{\textcolor{dkgreen}{VowelNotFound}}
    پاس خواهند شد.
    \\
    \\
    \\
    \\
    