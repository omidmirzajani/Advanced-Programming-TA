
\section{\rl{روند بازبینی کد برای درس برنامه‌سازی پیشرفته}}
با توجه به اهمیت بازبینی کد در آموزش در این درس از ابتدای ترم بازبینی کد بخشی از روند یادگیری برای شما می‌باشد. برای آشنایی با مراحل اضافه کردن بازبیننده کد به پروژه
\lr{AP98992}
در سایت
\lr{\href{https://dev.azure.com/}{Azure DevOps}}
به مراحل ذکر شده در ضمیمه
\ref{sec:add_reviewer}
مراجعه کنید.
\subsection{\rl{مرحله‌ی اول: تحویل تمرین}}
\begin{itemize}
	\item
	      تحویل اولیه تمرین با کامل شدن
	      \grayBox{Pull Request}
	      به
	      \grayBox{master}
	      انجام می‌شود. دقت کنید که هنگام کامل کردن
	      \grayBox{Pull Request}
	      گزینه
	      \grayBox{"Delete Source Branch"}
	      انتخاب نشده باشد.
	\item
	      برای اطمینان از کامل شدن درست
	      \grayBox{Pull Request}
	      در کامپیوتر خود دستورات زیر را اجرا کنید و از وجود پروژه خود در شاخه
	      \grayBox{master}
	      اطمینان حاصل کنید.
	      \begin{latin}
		      \begin{lstlisting}[frame=tb,basicstyle={\small\ttfamily},keywordstyle=\color{blue},keywords={checkout, pull}]
  git checkout master
  git pull
			\end{lstlisting}
	      \end{latin}
	\item
	      چنانچه به اشتباه گزینه
	      \grayBox{Delete Source Branch}
	      را انتخاب کرده باشید، لازم است با دستور زیر تمام شاخه‌های محلی را بار دیگر را روی سرور
	      \grayBox{push}
	      کنید.
	      \begin{latin}
		      \begin{lstlisting}[frame=tb,basicstyle={\small\ttfamily},keywordstyle=\color{blue},keywords={checkout, pull, push, origin}]
  git push --all origin
			\end{lstlisting}
	      \end{latin}
\end{itemize}
\subsection{\rl{مرحله دوم: بازبینی کد}}
\begin{itemize}
	\item
	      مهلت بازبینی کد یک هفته بعد از تحویل تمرین می‌باشد.
	\item
	      اولین مرحله مطلع‌کردن بازبیننده کد از اتمام تمرین و تقاضای بازبینی کد می‌باشد.
	\item
	      سپس بازبیننده کد بعد یک روز نتیجه بازبینی کد را در قالب کامنت‌ها لازم و نمره اولیه تمرین منعکس می‌کند.
	      \begin{latin}
		      \begin{itemize}
			      \item
			            Grade1://<AssignmentNumber>/<FirstGrade>
			      \item
			            e.g., Grade1://A1/90
		      \end{itemize}
	      \end{latin}
	\item
	      بعد از گرفتن کامنت‌ها، دانشجویان یک روز فرصت دارند تغییرات لازم در کد را برای جلب نظر بازبیننده کد در همان شاخه تمرین اعمال کرده و
	      \grayBox{Pull Request}
	      جدید با  نام
	      \grayBox{Pull Request}
	      اصلی/قبلی به‌علاوه‌ی پسوند
	      \grayBox{(\_review)}
	      برای فرستادن به
	      \grayBox{master}
		ایجاد کنند.
		مثال:
		\begin{latin}		
			\begin{table}[ht]
				\centering
				\begin{tabular}{c|c}
				\rowcolor[HTML]{CBCEFB} 
				Pull Request name before review & Pull Request name after review \\ \hline
				\rowcolor[HTML]{EFEFEF} 
				\texttt{HW3} & \texttt{HW3\textcolor{magenta}{\_review}} \\ \hline
				\texttt{HW4} & \texttt{HW4\textcolor{magenta}{\_review}} \\ \hline
				\rowcolor[HTML]{EFEFEF} 
				\texttt{HW5} & \texttt{HW5\textcolor{magenta}{\_review}}
				\end{tabular}
				\end{table}
		\end{latin}

	\item
	      این
	      \grayBox{Pull Request}
	      تا پایان مهلت بازبینی کد بازمانده و روند بازبینی کد با مهلت ۲۴ ساعت برای پاسخ از هر دو طرف ادامه پیدا می‌کند. در نهایت نمره باز‌بینی شده به صورت زیر در
	      \grayBox{Pull Request}
	      جدید منعکس می‌شود. نمره نهایی شما بر اساس نمره
	      \grayBox{Pull Request}
	      اولیه و
	      \grayBox{Pull Request}
	      جدید محاسبه خواهد شد.
	      \begin{latin}
		      \begin{itemize}
			      \item
			            Grade2://<AssignmentNum>/<FinalGrade>/<TotalComments>/<AddressedComments>
			      \item
			            e.g., Grade2://A1/90/10/5
			      \item
			            \begin{persian}
				            عدد
				            \grayBox{AddressedComments}
				            نشان‌گر تعداد کامنت‌هایی است که دانشجو مطابق نظر و رضایت بازبیننده کد تغییرات لازم را انجام یا جواب مناسب داده باشد.
			            \end{persian}
		      \end{itemize}
	      \end{latin}
	\item
	      در نهایت
	      \grayBox{Pull Request}
	      دوم شما باید قبل از اتمام مهلت بازبینی کد کامل شده و در
	      شاخه‌ی
	      \grayBox{master}
	      ادغام
	      (\grayBox{merge})
	      شود.
	\item
	      نمره نهایی شما بر اساس میزان مشارکت فعال شما در روند بازبینی‌کد خواهد بود.
\end{itemize}
\subsection{\rl{الگوهای ارزیابی}}
نمره تمرین‌ها بر اساس جدول زیر محاسبه می‌شود. این جدول به تناسب تمرین در مستند تمرین بروز خواهد شد.
\begin{latin}
	\begin{table}[ht]
		\centering
		\begin{tabular}{|c|c|c|c|c|c|c|}
			\hline
			\rowcolor[HTML]{FFFFC7}
			Subject                       & \begin{tabular}[c]{@{}c@{}}Program \\ logic\end{tabular} & \begin{tabular}[c]{@{}c@{}}Unit \\ tests\end{tabular} & \begin{tabular}[c]{@{}c@{}}Meaningful variable \\ and method names\\  and comments \\ where necessary\end{tabular} & \begin{tabular}[c]{@{}c@{}}Observe coding\\  conventions in \\ the names of \\ variables and \\ methods\end{tabular} & no redundancy & \begin{tabular}[c]{@{}c@{}}small \\ methods\end{tabular} \\ \hline
			\cellcolor[HTML]{FFFFC7}Grade & 30                        & 30                        & 10                        & 10                         & 10            & 10                         \\ \hline
		\end{tabular}
	\end{table}
\end{latin}
\textbf{توجه:}
نمره‌ی قسمت‌های
\lr{Program logic}
و
\lr{Unit tests}
فقط در صورتی داده می‌شود که برای شاخه‌ی
\grayBox{master}
راستی‌آزمایی ساخت
\LTRfootnote{Build validation}
فعال باشد.
هم‌چنین درصورت عدم قرارداد‌های نام‌گذاری برای شاخه‌ها، دایرکتوری و... نمره کسر خواهد شد.
