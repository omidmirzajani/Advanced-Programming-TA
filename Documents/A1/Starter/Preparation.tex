   \section{آماده سازی های اولیه}
    
        \subsection{ساخت پروژه ی C\#}
        برای ایجاد پروژه C\# کافی است کد زیر را در ترمینال خود اجرا کنید: 
        \LTR
        \begin{lstlisting}
        mkdir cs
        cd cs
        dotnet new sln
        mkdir cs
        cd cs
        dotnet new console
        cd ..
        dotnet sln add cs\cs.csproj
        mkdir cs.Tests
        cd cs.Tests
        dotnet new mstest
        dotnet add reference ..\cs\cs.csproj
        cd ..
        dotnet sln add cs.Tests\cs.Tests.csproj
        
        \end{lstlisting}
    
        \RTL
        
        \subsection{قواعد نام گذاری}
        
            قواعد نام‌گذاری تمرین را از جدول
            \ref{table:namingConvention}
            مطالعه کنید.
            \begin{table}[ht]
        	\centering
        	\caption{\rl{قراردادهای نام‌گذاری تمرین}}\label{table:namingConvention}	\begin{latin}
        	\begin{tabular}{|c|c|c|c|c|c|}
        		\hline
        		\rowcolor[HTML]{9698ED} 
        		\multicolumn{3}{|c|}{\cellcolor{mygray} Naming conventions}             \\ \hline 
        		Branch & Directory & Pull Request \\ \hline
        		
        		\texttt{fb\_A1} & \texttt{A1}        & \texttt{A1}          \\ \hline
        	\end{tabular}
            \end{latin}
            \end{table}
            \\
            \grayBox{\textcolor{blue}{*}}
            در کل یک دیرکتوری به نام A1 بسازید و داخل آن چهار دیرکتوری به نام های cs , java , python , cpp داشته باشید و فایل های مربوط به هر زبان را داخل دیرکتوری مربوطه بگذارید.
            