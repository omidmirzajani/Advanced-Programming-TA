\section{\rl{فرهنگ بازبینی کد}}
بازبینی کد در بسیاری از شرکت‌ها منشا کدورت و از هم پاشیدگی تیم‌های نرم‌افزاری می‌شود. جلوگیری از این مساله نیاز به فرهنگ‌سازی در جامعه نرم‌افزاری دارد. بهترین زمان و مکان برای شروع این فرهنگ‌سازی دوران تحصیل دانشجویان در دانشگاه است. از این جهت که:
\indent
\begin{enumerate}
	\item
	      این محیط عاری از رقابت‌های درون شرکتی برای ترفیع و برتری است.
	\item
	      دانشجو برای یادگیری در دانشگاه است و نسبت به بازخوردها کمتر جنبه دفاعی خواهد داشت.
\end{enumerate}
\noindent

\subsection{\rl{نکات مورد توجه}}
رعایت نکات زیر برای برگزاری یک بازبینی کد موثر لازم است:
\subsubsection{\rl{جدا نگه‌داشتن جنبه‌های شخصی از جنبه‌های فنی}}
بازخوردی که برای کد شما گذاشته می‌شود برای «کد» شماست، نه «شما». برداشت شخصی نکنید. هم‌چنین برای بازبیننده:‌ بازخورد شما باید متوجه «کد» باشد نه نویسنده کد.
\subsubsection{\rl{فرهنگ انتقادپذیری}}
همه ما می‌توانیم برنامه‌نویس‌های بهتری بشویم. بازبینی کد شما بهترین راه پیدا کردن فرصت‌های بهتر شدن است. از این جهت که شما سعی خود را برای ارائه بهترین کار خود در محدودیت زمانی که داشته‌اید کرده‌اید و بازبینی کد شما فرصت‌های بهتر شدن را به شما نشان می‌دهد.
\subsubsection{\rl{فرهنگ انتقاد سازنده}}
بازبینندگان کد هم لازم است علاقه به پیشرفت و بهتر شدن کد و برنامه‌نویس را در بازخوردها و لحن آن رعایت کنند. علاوه بر این سطح برنامه‌نویس را در بازخوردهای خود در نظر بگیرید. مثلا تذکرهای بسیار ظریف را به برنامه‌نویس مبتدی ندهید. و سلسله مراتب بازخوردها را رعایت کنید. یعنی اول بازخوردهای اساسی‌تر و مهم‌تر بعد بازخورد‌های جزئی‌تر و ظریف‌تر.
\subsubsection{\rl{فرهنگ شکرگزاری}}
بازبیننده کد از وقت و کار خود گذشته که کد شما را بررسی کند و راه‌های بهبود کد شما را به شما نشان بدهد. این بهترین راه یادگیری برای شماست. شکرگزار این فرصت باشید و در پیام‌های خود لحن مناسب بکار ببرید.
\subsubsection{\rl{صراحت در بازخورد}}
برای جلوگیری از پیام‌های بدون هدف و اتمام به موقع بازبینی کد، در بازخورد خود صراحت داشته باشید به طوری‌که بازبین‌شونده بتواند آن را در یکی از طبقه‌بندی‌های زیر جای دهد و تکلیف خود را بداند.
\begin{enumerate}[label=\Alph*)]
	\item
	      کد به صورت فعلی قابل قبول نیست و باید تصحیح شود.
	\item
	      کد به صورت فعلی قابل قبول است ولی اگر تصحیح شود بهتر است.
	\item
	      کد به صورت فعلی قابل قبول است ولی سلیقه من این است که تغییر کند.
	\item
	      چه کد جالبی. این رو قبلا نمی‌دونستم…

\end{enumerate}
\subsubsection{\rl{صراحت در قبول یا رد بازخورد}}
چنانچه با بازخورد موافق یا مخالف هستید، لازم است نظر موافق یا مخالف شما به صراحت در جواب شما با ذکر دلیل بیان شود.
