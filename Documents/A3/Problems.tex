 \section{
        رمزگشایی
    }
    \subsection{مشاهده فیلم}
    \RTL
    این تمرین با بقیه تمرین‎ها کمی فرق دارد. قبل از اینکه ادامه مستند را ببینید، 
    \textbf{حتماً}
    فیلم 
    \grayBox{\textcolor{blue}{The Imitation Game}}
    \RTL
    (بازی تقلید) را ببینید.
    و پیشنهادم به شما این است که تا فیلم را ندیدید، بقیه متن مستند را نخوانید.
    \\
    \subsection{قدرت کامپیوتر}
    همانطور که در فیلم دیدید، برای رمزگشایی یک متن ساده سال ها زمان میبرد و این کار در زمان جنگ اصلاً منطقی نیست.
    فرض کنید سرزمین ما ، هوا
    \footnote{HAVA}
    ، در حال جنگ با دشمنان است
    و هر لحظه خطر ما را تهدید میکند و دشمن به رهبری wizard در همه جای کشور نفوذ کرده است.
    آنها هر روز بین 6 تا 8 صبح به نیروهای خود در سراسر کشورمان پیام رادیویی مخابره میکنند و نیروهای جنگی آنها با توجه به محتویات پیام به نقاط مختلف کشور حمله میکنند.
    \\
    آنها این پیام ها را ابتدا توسط دستگاهی به نام انیگما
    \footnote{Enigma}
    کدگذاری میکنند تا در مسیرِ رسیدن به نیروهایشان، کسی از جزئیات
    حمله خبردار نشود.
    این پیام های رادیویی از طرف ما شنود شده است، اما کسی نمیداند محتویات این پیام ها چیست؟
    \\
     \begin{figure}[h!]
        \centering
        \includegraphics[width=1\linewidth]{images/map_Final.jpg}
        \caption{نقشه کشور}
        \label{fig:merge}
    \end{figure}\\
    حال ما برای جلوگیری از حملات دشمن به کمک شما نیاز داریم تا پیام های دشمن که در ماه February شنود شده است، را رمزگشایی کنید.
    این پیام ها که در پوشه Input
    قابل مشاهده است، شامل اطلاعاتی اعم از :
    \\
    1- زمان ارسال پیام
    \\
    2- نام ارسال کننده پیام
    \\
    3- هدف مورد نظر دشمن برای حمله
    \\
    4- نیروهای دشمن برای حمله
    \\
    5- ...
    \\
    است که توسط جاسوسان ما کشف شده است.
    \\
    \newpage
    همچنین نیروهای ما توانسته اند با سختی بسیار طی مدت طولانی فقط دو پیام از پیام های دشمن را به طور کامل رمزگشایی کنند که در زیر متن آن را مشاهده میکنید. امیدواریم در شکست دشمن به شما کمک کند.
    \\
     \begin{figure}[h!]
        \centering
        \includegraphics[width=1\linewidth]{images/texts.png}
        \caption{پیام نمونه}
        \label{fig:merge}
    \end{figure}\\
    گفته میشود نحوه کدگذاری ماشین انیگما به این صورت است که هر حرف از 26 حرف انگلیسی را به یک حرف دیگر،به طور تصادفی، تناظر میدهد. و این تناظر در طول یک نامه ثابت است.
    برای مثال حرف b را به z
    ،
    o را به a
    و
    k را به c تناظر میدهد.
    و اینگونه کلمه book به کلمه zaac تبدیل میشود.
    \\
    هم چنین اعداد به این گونه رمزگذاری میشوند که تمام اعداد(از 0 تا 9) جابجا(shift) میشوند.
    برای مثال اگر بدانیم 4 به 2 متناظر شده، حتماً 5 به 3 و 6 به 4 متناظر شده است.
    و اینگونه عدد 725 به صورت 503 کدگذاری خواهد شد.
    و این ترتیب نیز در طول یک نامه ثابت خواهد بود.
    \\
    جاسوسان ما اخیراً خبرهایی برایمان آوردند که شاید در حل مساله انیگما به شما یاری دهد:
    \\
    1- فقط حروف انگلیسی و اعداد کدگذاری میشوند.
    \\
    2- دسته های پیاده نظام دشمن گروه های 10 تایی است، لذا همیشه 10 تا 10 تا اعزام میشود.
    \\
    3- پیام ها را فقط فرمانده های کل ارسال میکنند.
    \\
    4- هدف مورد نظر دشمن برای حمله یکی از جاهایی است که در نقشه مشخص شده است.
    \\
    5- انیگما نسبت به حروف کوچک و بزرگ حساس نیست؛ یعنی اگر b به q متناظر شده است، B نیز به Q متناظر خواهد شد.
    \\
    با اطلاعات داده شده سعی کنید داده های داخل دیرکتوری Input را رمزگشایی کنید و سپس بخش بعدی را حل کنید.
    
    \subsection{تحلیل نامه ها}
    حال که به متون نامه ها دسترسی پیدا کردیم از شما میخواهیم به فرماندهان ما کمک کنید تا نیروهای مخصوص هر بخش را اعزام کند.
    \\
    تعداد نیروهای هر بخش اینگونه محاسبه میشود:
    \\
    1- 1.1 برابر تعداد پیاده نظام
    \footnote{infantry}
    های دشمن
    \\
    2- 2 برابر تعداد تانک
    \footnote{panzer}
    های دشمن
    \\
    3- 3 برابر تعداد هوپیماهای جنگی
    \footnote{bomber}
    دشمن
    \\
    به سلیقه خود و با طرز نگارش خود برای هر نامه از دشمن، و با توجه به اعداد ارقام، به فرمانده کل کشور پیامی ارسال کنید که او را از تعداد نیروهای لازم باخبر کند.
    این پیام ها باید به ترتیب شماره در دیرکتوری Output قرار گیرند.
    برای مثال آخر کار پوشه دیرکتوری شما باید شامل 30 فایل با فرمت .txt باشد
    که نام آنها Text1.txt 
    و
    Text2.txt
    و ...
    Text30.txt
    خواهد بود.
    \subsection{دیکشنری}
    بعضی کلمات و ترجمه آنها را در زیر میتوانید ببینید که ممکن است در نوشتن نامه به فرمانده به شما کمک کند:
    \\
    جزیره = 
    Land
    \\
    فرودگاه =
    Airport
    \\
    منطقه مسکونی =
    Area Residental
    \\
    پادگان نظامی =
    Base Millitary
    \\
    راه فرعی =
    Byway
    \\
    فرمانده کل =
    Commander-in-chief
    \\
    \\
    
    