
    \section{
    لپ تاپ مناسب!
    \grayBox{\textcolor{purple}{C\#}}
     \grayBox{\textcolor{green}{Java}}
    }
    دیمو
    \footnote{Dimo}
    به تازگی به کامپیوتر علاقه پیدا کرده و قصد دارد خود را برای مسابقات برنامه نویسی آماده کند.
    برای دیمو فقط ویژگی های 
    حافظه،
    پردازنده 
    و گرافیک از یک لپتاپ مهم است.
    او حافظه
    (Memory) 
    یک لپ تاپ را مهم ترین قسمت آن میداند.
    حال برای شروع کار، میخواهد او را در خرید لپ تاپ کمک کنید و متد 
    \grayBox{\textcolor{blue}{ChooseBest}}
    را پیاده سازی کنید.
    \subsection{پیاده سازی C\#}
    \RTL
    یک 
    enum
    به نام 
    Config
    تعریف کنید که شامل مقادیر 
    Graphic
    ،
    Memory
    و
    Cpu
    است.
    میتوانید از قطعه کد زیر به استفاده کنید:
    \LTR
    \begin{lstlisting}
      public enum Config
      {
          Graphic=/*TODO*/,
          Memory =/*TODO*/,
          Cpu =/*TODO*/
      }
    \end{lstlisting}
    \RTL
    متد 
    \grayBox{\textcolor{blue}{ChooseBest}}
    را در کلاس 
    Program
    قرار دهید 
    که ورودی آن یک 
    Config 
    باشد و خروجی مناسب داشته باشد.
    همانطور که گفتیم حافظه برای دیمو از هر چیزی مهم تر است؛
    \begin{itemize}
      
        \item 
    اگر این لپتاپ دارای هر سه ویژگی گرافیک، حافظه و پردازنده بود، 
     \grayBox{Excellent} 
    را خروجی دهد.
    
        \item
        
    اگر حافظه را به همراه فقط یکی از پردازنده و گرافیک داشت
    ،
    \grayBox{Very Good} 
    را خروجی دهد.
    
    \item
        
    اگر فقط حافظه داشت
    ،
    \grayBox{Good}  
    را خروجی دهد.
    
    \item
        
    اگر فقط گرافیک داشت
    ،
    \grayBox{Not Bad} 
    را خروجی دهد.
    
    \item
        
    اگر فقط پردازنده داشت
    ،
    \grayBox{Not Bad} 
    را خروجی دهد.
    \end{itemize}
    پس از پیاده سازی صحیح، تست 
    \grayBox{\textcolor{dkgreen}{ChooseBest\_Tests}}
    پاس خواهد شد.
    \\
    \subsection{پیاده سازی Java}
    مانند پیاده سازی سی شارپ 
    یک enum به نام Config با همان ویژگی های مذکور به علاوه سازنده اش تعریف کنید که آن را در متغیر 
    ConfigValue
    ذخیره کند.
    هم چنین متد 
    \grayBox{\textcolor{blue}{ChangeValue}}
    را به گونه ای پیاده سازی کنید که یک ورودی از نوع 
    int
    بگیرد و مقدار Value 
    را با آن برابر قرار دهد.
    برای مفهوم شدن سوال، میتوانید از قطعه کد زیر استفاده کنید:
    \LTR
    \begin{lstlisting}
      public enum Config 
      {
          Graphic(/*TODO*/),
          Ram(/*TODO*/),
          Cpu(/*TODO*/);
            
          public int ConfigValue;
          public int Value;
            
          private Config(int configValue)
          {
              this.ConfigValue = configValue;
          }
    
          //Implement ChangeValue
      }

    \end{lstlisting}
    \RTL
    پس از پیاده سازی صحیح سازنده و این متد، تست 
     \grayBox{\textcolor{dkgreen}{Config\_Test}}
     پاس خواهد شد.
    
    حال متد 
    \grayBox{\textcolor{blue}{ChooseBest}}
    را در کلاس 
    App
    بگونه ای پیاده سازی کنید  
    که ورودی آن یک 
    Config 
    باشد و خروجی مناسب داشته باشد. 
    
    
    پس از پیاده سازی صحیح، تست 
    \grayBox{\textcolor{dkgreen}{ChooseBest\_Test}}
    پاس خواهد شد.
    \section{جابجایی لپتاپ!}
    دیمو که از خرید لپتاپ جدیدش خیلی راضی نشده، تصمیم گرفته بعضی از اجزای لپتاپ جدیدش را تعویض کند و از اجزای لپتاپ قبلی خود استفاده کند.
    برای کمک به او ابتدا کلاس های زیر را پیاده سازی کنید؛
    \subsection{
    کلاس Graphic
    \grayBox{\textcolor{purple}{C\#}}
    }
    هر کارت گرافیک دارای ویژگی های 
    \begin{itemize}
        \item 
            \grayBox{Size}
             از نوع 
            int 
        \item
            
            \grayBox{Coprocessor}
            از نوع
            string
        \item
        
            \grayBox{Type}
            از نوع 
            int 
    \end{itemize}
    این کلاس را به همراه سازنده(Constructor) اش به گونه ای پیاده سازی کنید که همه 
    \grayBox{Property}
    ها، مقدار null را نیز بپذیرد.
    پس از پیاده سازی صحیح این کلاس،
    تست
    \grayBox{\textcolor{dkgreen}{GraphicConstructor\_Tests}}
    پاس خواهد شد.
    
    \subsection{
    کلاس Memory
    \grayBox{\textcolor{purple}{C\#}}
    }
    هر کارت حافظه دارای ویژگی های 
    \begin{itemize}
        \item 
            \grayBox{Capacity}
             از نوع 
            int 
        \item
            
            \grayBox{Pins}
            از نوع
            int
        \item
        
            \grayBox{Type}
            از نوع 
            int 
    \end{itemize}
    این کلاس را به همراه سازنده(Constructor) اش به گونه ای پیاده سازی کنید که همه 
    \grayBox{Property}
    ها، مقدار null را نیز بپذیرد.
    پس از پیاده سازی صحیح این کلاس،
    تست
    \grayBox{\textcolor{dkgreen}{MemoryConstructor\_Tests}}
    پاس خواهد شد.
     
    \subsection{
    کلاس Cpu
    \grayBox{\textcolor{purple}{C\#}}
    }
    هر پردازنده دارای ویژگی های 
    \begin{itemize}
        \item 
            \grayBox{Model}
             از نوع 
            string 
        \item
            
            \grayBox{Weight}
            از نوع
            double
        \item
        
            \grayBox{Speed}
            از نوع 
            string 
    \end{itemize}
    این کلاس را به همراه سازنده(Constructor) اش به گونه ای پیاده سازی کنید که همه 
    \grayBox{Property}
    ها، مقدار null را نیز بپذیرد.
    پس از پیاده سازی صحیح این کلاس،
    تست
    \grayBox{\textcolor{dkgreen}{CpuConstructor\_Tests}}
    پاس خواهد شد.
    
    \subsection{
    جابجایی
    \grayBox{\textcolor{purple}{C\#}}
    }
    حال که کلاس های 
    \grayBox{Graphic}
    ،
    \grayBox{Memory}
    و
    \grayBox{Cpu}
    را به درستی پیاده سازی کردید،
    به دیمو کمک کنید که اجزای لپتاپ جدیدش را با قبلی عوض کند.
    \\
    متد 
    \grayBox{\textcolor{blue}{SwapConfigs}}
    را در کلاس Program
    به گونه ای پیاده سازی کنید که دو object بگیرد و این دو را با یکدیگر عوض کند برای مفهوم تر شدن سوال به تست ها مراجعه کنید.
    پس از پیاده سازی صحیح این متد،
    تست
    \grayBox{\textcolor{dkgreen}{SwapConfigs\_Tests}}
    پاس خواهد شد.
    
    \newpage
    \section{ساختار حافظه!}
    دیمو به خاطر علاقه زیادش به حافظه، تصمیم گرفته به عنوان اولین تمرین برنامه نویسی خود، با مفاهیم آن کار کند.
    او می خواهد بداند که داده های ارجاعی 
    (\grayBox{Reference Type})
    و 
    مقداری
    (\grayBox{Value Type})
    چگونه حافظه را اشغال میکنند.
    
    \subsection{
        آشنایی با Stack
        \grayBox{\textcolor{purple}{C\#}}
    }
    نوع داده ای 
    \grayBox{\textcolor{blue}{Struct\_Size5}}
    را به گونه ای پیاده سازی کنید 
    که اندازه متغیر این نوع داده ای،
    5 
    باشد.
    \\
    سپس به طور مشابه 
    نوع های داده ای
    \grayBox{\textcolor{blue}{Struct\_Size10}}
    ،
    \grayBox{\textcolor{blue}{Struct\_Size12}}
    و
    \grayBox{\textcolor{blue}{Struct\_Size105}}
    را نیز پیاده سازی کنید.
    پس از پیاده سازی صحیح،
    تست
    \grayBox{\textcolor{dkgreen}{StructSize\_Tests}}
    پاس خواهد شد.
    
    \subsection{
        آشنایی با Heap
        \grayBox{\textcolor{purple}{C\#}}
    }
    کلاس
    \grayBox{MemoryHeap}
    را به همراه متدهایش
    پیاده سازی کنید.
    \\
    متد 
    \grayBox{\textcolor{blue}{Allocate}}
    یک عدد میگیرد 
    و به اندازه آن
    (با واحد 
    byte
    )
    حافظه را اشغال میکند.
    \\
    متد 
    \grayBox{\textcolor{blue}{DeAllocate}}
    که با فراخوانی آن حافظه که اشغال شده را خالی میکند.
    برای مفهوم شدن سوال به تست ها مراجعه کنید.
    پس از پیاده سازی صحیح،
    تست
    \grayBox{\textcolor{dkgreen}{HeapSize\_Tests}}
    پاس خواهد شد.
    \\
    \\
    \\
    \\
    