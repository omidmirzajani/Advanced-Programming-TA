
    \section{آماده سازی های اولیه}
    
        \subsection{ساخت پروژه ی C\#}
        برای ایجاد پروژه C\# کافی است کد زیر را در ترمینال خود اجرا کنید: 
        \LTR
        \begin{lstlisting}
        mkdir A3_cs
        cd A3_cs
        dotnet new sln
        mkdir A3_cs
        cd A3_cs
        dotnet new console
        cd ..
        dotnet sln add A3_cs\A3_cs.csproj
        mkdir A3_cs.Tests
        cd A3_cs.Tests
        dotnet new mstest
        dotnet add reference ..\A3_cs\A3_cs.csproj
        cd ..
        dotnet sln add A3_cs.Tests\A3_cs.Tests.csproj
        
        \end{lstlisting}
    
        \RTL
        
        \subsection{قواعد نام گذاری}
        
            قواعد نام‌گذاری تمرین را از جدول
            \ref{table:namingConvention}
            مطالعه کنید.
            \begin{table}[ht]
        	\centering
        	\caption{\rl{قراردادهای نام‌گذاری تمرین}}\label{table:namingConvention}	\begin{latin}
        	\begin{tabular}{|c|c|c|c|c|c|}
        		\hline
        		\rowcolor[HTML]{9698ED} 
        		\multicolumn{3}{|c|}{\cellcolor{mygray} Naming conventions}             \\ \hline 
        		Branch & Directory & Pull Request \\ \hline
        		
        		\texttt{fb\_A3} & \texttt{A3}        & \texttt{A3}          \\ \hline
        	\end{tabular}
            \end{latin}
            \end{table}
            \\
            \grayBox{\textcolor{blue}{*}}
            در کل یک دیرکتوری داخل Assignments به نام A3 بسازید و داخل آن،یک دیرکتوری به نام  A3\_cs داشته باشید و فایل های مربوط به آن را داخل دیرکتوری مربوطه بگذارید.
            
     
   